\chapter{INTRODUCTION}
%\chapter[short entry]{long title}
\label{chap:intro}
\section{Introduction}
Weather prediction is the task of predicting the atmosphere at a future time and a given area. This has been done through physical equations in the early days in which the atmosphere is considered fluid[1]. The current state of the environment is inspected, and the future state is predicted by solving those equations numerically, but we cannot determine very accurate weather for more than 10 days and this can be improved with the help of science and technology.
\\Machine learning can be used to process immediate comparisons between historical weather forecasts and observations. With the use of machine learning, weather models can better account for prediction inaccuracies, such as overestimated rainfall, and produce more accurate predictions. Temperature prediction is of major importance in a large number of applications, including climate-related studies, energy, agricultural, medical, or etc. 
\\There are numerous kinds of machine learning calculations, which are Linear Regression, Polynomial Regression, Random Forest Regression, Artificial Neural Network, and Recurrent Neural Network. These models are prepared depending on the authentic information given of any area. Contribution to these models is given, for example, if anticipating temperature, least temperature, greatest temperature, mean dampness, and order for 2 days. In light of this Minimum Temperature and Maximum Temperature of 7 days will be accomplished..
\\
\\
\\
\\
\\

\section{Purpose}
Every Human is subject  to  adjusting  themselves  with  respect  to  weather  conditions  for  their dressing habits to  strategic organizational  planning  activities,  since the  adverse weather conditions may cause considerable damage to lives and properties. We need to be on alert for these adverse weather conditions by taking some precautions and using prediction mechanisms to detect them and provide early warning of hazardous weather phenomena. Weather prediction is an indispensable requirement for all of us. Weather is important for most aspects of human life.  Predicting weather is very useful.  Humans have attempted to make predictions about the weather, many early religions used gods to explain the weather.  Only relatively recently have humans developed reasonably accurate weather predictions.    We decided to collect weather  data  and  measure  the accuracy of predictions made using linear regression.The Weather prediction model designed by us would be of great use to the farmers and for normal beings as well. In temperature forecasting one has to distinguish between the times the forecast goes ahead, for example temperature one hour ahead or minimum and maximum temperature of a given day. Several works has been done and different artificial neural networks (ANN) models have been tested[2]. The observations include: 
\\
\textbf{Temperature -}
the measure of warmth or coldness
\\
\textbf{Humidity-}
the amount of moisture in the atmosphere
\\
\textbf{Precipitation -}
the amount of moisture(usually rain or snow) which falls in the ground
\\
\textbf{Pressure -}
the force atmosphere applies on the environment
\\
\\
\\
\\


% \begin{equation}
%    BW_p=\frac{f_h - f_l}{f_c} * 100\%
% \end{equation}
% Broadband by the ratio :
% \begin{equation}
%    BW_b=\frac{f_h}{f_l}
% \end{equation}
% where, $ f_c$ is Center frequency,$f_l$ is Higher Cut-off frequency and $ f_h $ is lower Cut-off frequency
\section{Machine Learning }
Machine Learning  is  the  study  of  computer  algorithms  that  improve  automatically through experience. Applications range from data mining programs that discover general rules in large  data  sets,  to  information  filtering  systems  that  automatically  learn  users' interests. 
\\
Machine Learning is concerned with the design and development of algorithms that allow computers to  evolve  behaviors  based  on  empirical  data,  such  as  from  sensor  data  or databases[2].  A major  focus  of  Machine  Learning  research  is  to  automatically  learn  to recognize complex  patterns  and make intelligent  decisions  based on  data;  the difficulty lies  in  the  fact  that  the  set  of  all  possible  behaviors  given  all  possible  inputs  is  too complex to describe generally in programming languages, so that in effect programs must automatically describe programs.
\\
The poor performance results produced by statistical estimation models have flooded the estimation area for over the last decade. Their inability to handle categorical data, cope with  missing data  points,  spread  of data  points  and  most  importantly  lack  of  reasoning capabilities  has  triggered  an  increase  in  the  number  of  studies  using  non-traditional methods  like  machine  learning  techniques.  The  area  of  machine  learning  draws  on concepts  from  diverse  fields  such  as  statistics,  artificial  intelligence,  philosophy, information  theory,  biology,  cognitive  science,  computational  complexity  and  control theory[3].
\\
\\
\\
\\
\\

% Typically, the return loss is determined as follows:
% \begin{equation}
%     R = 10\log_{10}(\frac{Pi}{Pr})
% \end{equation}
% where $P_i$ is the forward power and $P_r$ is the reflected power 
% when reflection coefficient $( \Gamma )$ = $\sqrt{\frac{P_r}{P_i}}$  , ${P_r}$ is reflected power and $P_i$ is forward power. 
% Thus,
% \begin{equation}
% R(dB) =  -20\log_{10}(\Gamma)
% \end{equation}
% when utilizing a vector network analyzer to measure a network's properties. The return loss is a crucial characteristic as a result.

% \subsection{Voltage standing wave ratio }

% According to the voltage standing wave ratio (VSWR) definition, the VSWR is determined by dividing the line's maximum voltage by its minimum voltage. The sum of the voltage components from the forward power and the reflected power causes the voltage variations.
% \begin{equation}
%     VSWR= \frac{Vmax}{Vmin}
% \end{equation}
% VSWR stated in terms of the voltages for the forward and reflected waves.\\
% \begin{equation}
%     VSWR= \frac{Vfwd+Vref}{Vfwd-Vref}
% \end{equation}
% VSWR as a function of the reflection coefficient. 
% \begin{equation}
%     VSWR =\frac{(1+|\Gamma| )}{(1-|\Gamma|)}
% \end{equation}
% where, $\Gamma$ = reflection coefficient, the reflection coefficient, $\Gamma$ is defined as the ratio of the reflected current or voltage vector to the forward current or voltage.
% \subsection{Input Impedance}

% For maximum power transfer in an antenna, we need to keep the VSWR (voltage standing wave ratio) as low as possible, which can only be done when we achieve impedance matching. For this purpose, we need to take input impedance carefully into account. It is the total sum of resistance and reactance at the input terminal.

% \subsection{Resonant frequency }

% The resonant frequency is the frequency at which the patch receives the most power, i.e., when the feedline and patch's impedances are most closely matched. Additionally, it is the point at which the impedance is solely resistive and inductive reactance equals capacitive reactance\cite{kraus2006antennas}.

% \subsection{Radiation Pattern}

% The radiation pattern is the pictorial representation of antenna radiation. When the antenna radiates, then radiation lobes are created, which consist of some prominent lobes called the main lobe or major lobe, and various side lobes or minor lobes. These lobes account for the directivity of the antenna as to which direction the antenna radiates the most\cite{kraus2006antennas}.

% \subsection{Half Power Beam width}
% Half Power Beam width (HPBW) is evaluated by the angular width of the major lobes at $ \log \frac{1}{\sqrt 2} $ points (or $-3db$ points)\cite{balanis2015antenna}.

% \subsection{Polarization}
% When an antenna radiates electromagnetic waves, an electric field is induced, and the orientation of such an induced electric field is called polarization in the antenna. There can be broadly three kinds of polarization, namely, linear, circular, and elliptical\cite{kraus2006antennas}. 

% \subsection{Gain}
% “The ratio of radiation intensity in one direction to that which would be obtained if the antenna were to broadcast its power isotopically.”\cite{balanis2015antenna} 
% \\it is given as 
% \begin{equation}
% G = \frac{U(\theta,\phi)}{U_a} = \frac{4\phi U (\theta,\phi)}{P_a}
% \end{equation}
% where $P_a$ is the input power to the antenna.
% An antenna's efficiency determines its gain. \\(Gain = Directivity X radiation efficiency)\cite{pues1989impedance}
% The gain of an antenna can also be expressed in terms of its effective aperture$A_e$ and wavelength. 
% \\This is,  
% \begin{equation}
%    G=\frac{4\pi}{\lambda^2}A_e
% \end{equation}
% \subsection{Directivity}

% The maximum gain in a specific or intended direction is referenced as directivity. 
% \subsection{Efficiency}

% "The proportion of an antenna's radiated power to its input power is known as antenna efficiency." Antenna efficiency ($\eta_{ant}$) and Radiation efficiency ($\eta_{rad}$) are both calculated in percentage terms ( \%).
% \\Below is a mathematical expression for antenna efficiency.
% \begin{equation}
%  \eta_{ant} = \frac{P_{rad}}{P_{input}}   
% \end{equation}
% where,
% $\eta_{ant}$ is antenna efficiency, $P_{rad}$ is the power radiated, $P_{input}$ is the input power.
\section{Types of Machine Learning}

There  are  two  main  types  of  Machine  Learning  algorithms.  In  this  project,  supervised learning  is  adopted  here  to  build  models  from  raw  data  and  perform  regression  and classification.
\\
\\\textbf{Supervised Learning-}Supervised Learning is a machine learning paradigm for acquiring the input output relationship information of a system based on a given set of paired input- output  training  samples.  As  the  output  is regarded  as  the  label  of  the  input  data  or  the supervision,  an  input-output  training  sample  is  also  called  labeled  training  data,  or supervised data. Learning from Labeled Data, or Inductive Machine Learning.    The goal of supervised learning is to build an artificial system that can learn the mapping between  the  input  and  the  output,  and  can  predict  the  output  of  the  system  given  new inputs. If the output takes a finite set of discrete values that indicate the class labels of the input,  the  learned  mapping  leads  to  the  classification  of  the  input  data.  If  the  output takes continuous values, it leads to a regression of the input. It deduces a function from training data that maps inputs to the expected outcomes. The output of the function can be a predicted continuous value (called regression), or a predicted class label from a discrete set  for  the  input  object  (called  classification).  The  goal  of  the  supervised  learner  is  to predict  the  value  of  the  function  for  any  valid  input  object  from a  number  of  training examples.  The  most  widely  used  classifiers  are  the  Neural  Network  (Multilayer perceptron),  Support  Vector  Machines,  Regression Analysis, Artificial neural networks and time series analysis[4].
% . Four main categories can be used to classify all microstrip antennas are given in Figure \ref{Fig 1.2 Basic Categories of MSAs}
\\
\\\textbf{Unsupervised Learning-}
Unsupervised learning studies  how  systems  can  learn  to represent  particular  input  patterns  in  a  way  that  reflects  the  statistical  structure  of  the  overall collection  of  input  patterns.  By contrast  with  supervised  learning  or  reinforcement learning, there are no explicit target outputs or environmental evaluations associated with each input; rather the unsupervised learner brings to bear prior biases as to what aspects of the structure of the input should be captured in the output.


 %The fractional operator is represented by
 %$  _a D_t ^\alpha f(t)$ with $ t,a\in %\textbf{R} $ where $ t>a $%
% \begin{figure}[htpb]
% 	\begin{center}
% 		\includegraphics[width=0.9\linewidth]{images/C_1/Fig 1.3 Microstrip antenna configuration.png}
% 		\caption{Microstrip antenna configuration}
% 		\label{Fig 1.3 Microstrip antenna configuration}
% 	\end{center}
% \end{figure}
% Additionally, the patch can be of various shapes and sizes as per the requirement of antenna design. For various shapes of patches see fig \ref{Fig 1.4 Various patch Shapes} \cite{chourasiacomparative}
% \begin{figure}[htpb]
% 	\begin{center}
% 		\includegraphics[width=0.9\linewidth]{images/C_1/Fig 1.4 Various patch Shapes.png}
% 		\caption{Various patch Shapes}
% 		\label{Fig 1.4 Various patch Shapes}
% 	\end{center}
% \end{figure}

% \begin{figure}[htpb]
% 	\begin{center}
% 		\includegraphics[width=0.7\linewidth]{images/C_1/Fig 1.5  structure of RMSPA.png}
% 		\caption{Structure of RMSPA}
% 		\label{Fig 1.5 Structure of RMSPA}
% 	\end{center}
% \end{figure}

% MSPAs can be designed by taking a rectangular patch and FR-4 substrate with a worthwhile thickness whose dielectric constant lies in the range of 3.8. to 4.8 which is considered to be as low in value to reap the desired gain of the antenna leading to a substantially good performance refer figure \ref{Fig 1.5 Structure of RMSPA}.

% As the microstrip patch antenna is a resonant antenna, thereby feedlines become perhaps the most important aspect of an antenna as it allows it to drive on the desired frequency. We have several methods by which antenna can be fed. Most admired feeding techniques \cite{kraus2006antennas} are microstrip feedline, Aperture Coupling, Proximity Coupling, Coaxial Probe, and coplanar waveguide (CPW) feeding. Among the list, we choose that method with which we can derive maximum power transfer and minimum input loss.

% \subsection{Microstrip feedline}

% A microstrip feedline is a narrow conducting strip that is minuscule in terms of width as compared to that of the patch. The feed-line has the advantage that it can be easily constructed by just etching along with the patch of the antenna, giving rise to the planar structure. The feed locale can be maneuvered easily leading to flawless impedance matching\cite{kraus2006antennas}.


%  \subsection{Aperture Coupling}	
 
% The aperture technique for antenna feed is used when the antenna consists of two substrates (i.e., lower and upper substrate), provided that the ground plane is between the substrates and a microstrip feedline is underneath the lower substrate\cite{kumar2013performance}. 

%  \subsection{Proximity Coupling}
 
% Unlike the aperture technique for antenna feed, the ground plane lies underneath the two substrates. In proximity coupling the spurious radiation is low. Although this technique is easy to design, we encounter a lot of complications with the fabrication\cite{kraus2006antennas} \cite{kumar2013performance}. 

% \subsection{Coaxial Probe}

% In the coaxial probe technique for antenna feed, the coaxial probe's inner conductor is attached to a patch, allowing the element to radiate while the outside conductor makes contact with the ground plane. Since the intended location of the feed inside the patch can be easily matched to the input impedance, this technique is more extensively used. However, the construction of the antenna is a complicated job\citep{kumar2013performance}.

% \subsection{Coplanar Waveguide (CPW) feeding}

% The coplanar waveguide technique for antenna feed has a simple configuration in which there is no need for coupling via holes\cite{yahia2012b13}. 
\section{Problem Identification}
 Weather prediction is a useful tool for informing populations of expected weather conditions. Weather prediction is a complex topic and poses significant variation in practice. We will attempt to understand and implement a weather prediction application.
\\
\\
A detailed study of the process should be done with a variety of  techniques  such  as  interviews,  questionnaires  etc.  Data collected by these sources must be evaluated in order to reach a conclusion. The conclusion is to understand how the system works. This program is called the existing system. The current system  is now  being processed  and the problem  location is identified. The designer now acts as a problem solver and tries to solve the problems the business is facing[5]. Solutions are offered as suggestions. The proposal is then weighed against an existing system  by analysis  and  selection  is best.  The  suggestion is presented to the user for user approval.
\\
\\
Every  Human is subject  to  adjusting  themselves  with  respect  to  weather  conditions  for  their dressing habits to  strategic organizational  planning  activities,  since the  adverse weather conditions may cause considerable damage to lives and properties. 
\\
\\
\\
\\
\\
\\
\\
\\
\\
\\

\section{Use of Algorithms }
There are different methods of foreseeing temperature utilizing Regression and a variety of Functional Regression, in which datasets are utilized to play out the counts and investigation. To Train, the calculations 80 percent size of information is utilized and 20 percent size of information is named as a Test set. 
\\
\\
For Example, if we need to anticipate the temperature of Jhansi, India utilizing these Machine Learning calculations, we will utilize 8 Years of information to prepare the calculations and 2 years of information as a Test dataset. The as opposed to Weather Forecasting utilizing Machine Learning Algorithms which depends essentially on reenactment dependent on Physics and Differential Equations, Artificial Intelligence is additionally utilized for foreseeing temperature: which incorporates models, for example, Linear regression, Decision tree regression, Random forest regression[6].
\\
\\
To finish up, Machine Learning has enormously changed the worldview of Weather estimating with high precision and predictivity. What's more, in the  couple of years greater progression will be made utilizing these advances to precisely foresee the climate to avoid catastrophes like typhoons, Tornados, and Thunderstorms.
