\chapter{CONCLUSION}
These methods are extremely easy to adopt as they don’t require any specific deep technical concepts to be clear in.Nonetheless,\textbf{predictions perfectly fit in the error range designed by the dataset itself} . It is important to consider that we only have examined monthly average values while it may be interesting to consider daily values too and have daily predictions.
\\
\\
We have implemented the ARIMA model of weather forecasting on Jhansi’s weather dataset. The implementation uses the Forward Fill method in the data cleaning process to fill missing values. We proposed a modification in the data cleaning process- to fill missing values using the mean of the observations.  Auto ARIMA is then applied on the modified temperature Mean Squared Error(MSE) is used to evaluate the model's performance for a certain data cleaning method. For Non-stationary data, Mean of the Observations Data Cleaning process in Auto ARIMA forecasting is a better approach than Forward fill to fill missing values. All the machine learning models: linear regression,decision tree regression, random forest regression were beaten by expert climate determining apparatuses, even though the error in their execution reduced significantly for later days, our models may beat genius professional ones[19].
\\
\\
In the second model which we made was the Neural Network Model.Neural network is trained with seventy percent of the input data. Where the model is trained using this observed data to forecast the weather, followed by testing done using remaining thirty percentages of input data. Then the mean squared error and accuracy is calculated for the model by comparing the output of testing with target output. This model generates output in terms of minimum and maximum of various parameters.other machine-learning methods have been applied to various needs for targeted weather forecasts. Such applications include  forecasting for agricultural decision support , forecasting road weather to enhance the safety of surface transportation[20].