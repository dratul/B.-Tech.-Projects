% Abstract

\abstract
\noindent \textbf{KEYWORDS:} \hspace*{0.5em} \parbox[t]{4.4in}{Machine Learning, Models, Algorithms, Naive Bayes, NLP, Ubidots, CNN, ECG}

\vspace*{24pt}

\noindent The prevalence of depression as a global mental health disorder underscores the urgency for early detection and intervention to prevent its devastating consequences, including suicide. This project addresses this critical need through a novel approach leveraging multimodel sensing and machine learning techniques, notably the Naive Bayes algorithm and Natural Language Processing (NLP). By integrating speech-to-text technology and ECG analysis, users are prompted to respond to questions assessing their mental state. The collected data is analyzed using the Naive Bayes algorithm to predict the likelihood of depression. Additionally, a hardware component employing an ESP8266 and AD8232 ECG sensor monitors users heart rate, transmitting data to the cloud for remote assessment by medical professionals. Concurrently, the project tackles the challenge of accurately detecting emotions in speech through an Automated Speech Emotion Recognition system utilizing Convolutional Neural Network (CNN) algorithms. By combining these methodologies, the project aims to provide a comprehensive solution for depression detection and intervention, ultimately improving patient outcomes and alleviating the burden on healthcare systems. Additionally, the incorporation of sentiment analysis into the speech emotion recognition framework represents a significant advancement, offering valuable insights into individuals mental health states and enhancing depression detection methods.
\pagebreak